\documentclass[10pt,notheorems]{beamer}
\usetheme[progressbar=frametitle]{metropolis}
\usepackage{appendixnumberbeamer}

\usepackage{booktabs}
\usepackage[scale=2]{ccicons}
\usepackage{amsmath}
\usepackage{amsthm}
\setbeamertemplate{theorems}[numbered] % to number

\theoremstyle{plain} % insert bellow all blocks you want in italic
\newtheorem{theorem}{Theorem} 
\usepackage{amssymb}
\usepackage{mathtools}
\usepackage{tikz}
\usepackage{graphicx}
\graphicspath{{C:\Users\swost\Gdb\pds_def.jpg}}
\graphicspath{{C:\Users\swost\Gdb\Algo_WL.jpg}}
\graphicspath{{C:\Users\swost\Gdb\dcpds.jpg}}
\graphicspath{{C:\Users\swost\Gdb\op_semantics.jpg}}
\graphicspath{{C:\Users\swost\Gdb\Delta_t_defn.jpg}}
\usepackage{pgfplots}
\usepgfplotslibrary{dateplot}
\usepackage{cite}
\usepackage{todonotes}
\usepackage{upgreek}
\usepackage{bm}
\usepackage{xspace}
\newcommand{\themename}{\textbf{\textsc{metropolis}}\xspace}
\usepackage{subcaption}

\definecolor{CanvasBG}{HTML}{FAFAFA}

% From the official style guide
\definecolor{UnccGreen}{HTML}{00703C}
\definecolor{UnccGold}{HTML}{B3A369}
\definecolor{UnccLightGreen}{HTML}{C3D7A4}
\definecolor{UnccYellow}{HTML}{F0CB00}
\definecolor{UnccOrange}{HTML}{F3901D}
\definecolor{UnccLightYellow}{HTML}{FFF6DC}
\definecolor{UnccBlue}{HTML}{00728F}
\definecolor{UnccPink}{HTML}{DE3A6E}
\definecolor{White}{HTML}{FFFFFF}
\definecolor{LightGray}{HTML}{DDDDDD}

\definecolor{WarmGray}{HTML}{696158}
\definecolor{StoneGray}{HTML}{717C7D}
\definecolor{DarkGreen}{HTML}{2C5234}
\definecolor{LightGreen}{HTML}{509E2F}
\definecolor{BrightGold}{HTML}{F0CB00}

\definecolor{Royal}{HTML}{72246C}
\definecolor{Ocean}{HTML}{006BA6}
\definecolor{Flash}{HTML}{B52555}
\definecolor{Citrus}{HTML}{FFB81C}
\definecolor{Spring}{HTML}{CEDC00}

\definecolor{Garden}{HTML}{B7CE95}
\definecolor{Sand}{HTML}{F0E991}
\definecolor{Bloom}{HTML}{F1E6B2}
\definecolor{Clay}{HTML}{B7B09C}
\definecolor{Cloud}{HTML}{BAC5B9}

\setbeamercolor{frametitle}{bg=UnccGreen}
\setbeamercolor{progress bar}{bg=BrightGold, fg=UnccGreen}
\setbeamercolor{alerted text}{fg=Flash}

\setbeamercolor{block title}{bg=LightGreen, fg=White}
\setbeamercolor{block title example}{bg=Ocean, fg=White}
\setbeamercolor{block title alerted}{bg=Citrus, fg=White}
\setbeamercolor{block body}{bg=CanvasBG}

\metroset{titleformat=smallcaps, progressbar=foot}

\makeatletter
\setlength{\metropolis@progressinheadfoot@linewidth}{2pt}
\setlength{\metropolis@titleseparator@linewidth}{2pt}
\setlength{\metropolis@progressonsectionpage@linewidth}{2pt}

\usepackage[T1]{fontenc}
\usefonttheme{serif}
\usefonttheme{professionalfonts}
\usepackage{amsmath, tgpagella, eucal, eufrak}
\usepackage{amsfonts}
\usepackage{mathrsfs}
\usepackage{amssymb}
\usepackage{graphicx}
\usepackage{varioref}
\usepackage{hyperref}
\usepackage[noabbrev]{cleveref}
\DeclareMathOperator*{\argmin}{arg\,min}
\DeclareMathOperator*{\argmax}{arg\,max}
\title{Context-bounded Model Checking}
\subtitle{}
% \date{\today}
\date{}
\author{Om Swostik Mishra}
\institute{IIT Bombay}
% \titlegraphic{\hfill\includegraphics[height=1.5cm]{logo.pdf}}

\begin{document}

%insert frame on computing post*
%insert some frames on the paper mam sent i.e elaborate on the restriction caused by the assumptions and
%discuss the recent progress
%describe fork-join, don't just read slides
%let's make a txt file containing comments and ideas

\begin{frame}
    \maketitle
\end{frame}
\begin{frame}{Introduction}
    \begin{itemize}
        \nocite{qadeer2005context}
        \item The interaction between concurrently executing threads of a program might lead to programming 
        errors that are difficult to reproduce and fix.
        \item In general, the problem of verifying a concurrent Boolean program is undecidable \cite{ramalingam2000context}.
        \item We will show that the analysis is decidable if the number of context switches
        is bounded by an arbitrary constant.
        \item In general, restricting the number of context switches is not sound as there could be dynamic systems 
        with multiple processes and threads where context switches happen at a high frequency .
        \item However, the analysis is sound up till the context bound as every thread is explored for an 
        unbounded stack depth.
    \end{itemize}
\end{frame}
\begin{frame}{The Problem}
    \begin{itemize}
        \item We will analyze the following problem,\\
        Given a Boolean program $P$ and an integer $k$, does $P$ go wrong by failing an assertion with at most $k$
        contexts?
        \item A \textit{context} is an uninterrupted set of actions by a single thread.
        \item An execution with $k$ contexts invloves $k-1$ \textit{context switches} i.e execution 
        switches from one thread to another exactly $k-1$ times.
        \item The above problem is decidable and there exists an algorithm which decides it, that is polynomial in the size of 
        $P$ and exponential in the size of $k$.
    \end{itemize}
\end{frame}
\begin{frame}{Pushdown Systems}
    \vskip-1ex
    \begin{center}
    \includegraphics[scale=0.4]{pds_def.jpg}
    \end{center}
    \begin{itemize}
      \item A single-stack pushdown system is a 5-tuple, 
      \[P=\:(G,\Gamma,\Delta,g_{in},w_{in})\]
      \item $G$ and $\Gamma$ refer to the set of \textit{global states} and \textit{stack alphabet} respectively.
      \item A \textit{stack} $w$ is an element of $\Gamma^*$.
      \item A \textit{configuration} is a member of $G\times \Gamma^*$, such that any $c\in G\times \Gamma^*$ can be written as $c=\langle g,w \rangle$, where 
      $g\in G$ and $w\in \Gamma^*$.
    \end{itemize}
\end{frame}
\begin{frame}{Pushdown Systems (continued)}
    \begin{itemize}
        \item The \textit{transition relation} $\Delta$ over $G$ and $\Gamma$ is a finite subset
        of $(G\times \Gamma)\times(G\times \Gamma^*)$.
        \item $c_{in}=\langle g_{in},w_{in}\rangle$ refers to the initial configuration of 
        the pushdown system $P$.
        \item Given $\Delta$, we can define a transition system on configurations $\rightarrow_{\Delta}$ as follows:\\
        \[\langle g,\gamma w^{\prime} \rangle \rightarrow_{\Delta} \langle g^{\prime},ww^{\prime}\rangle,
        \:\:\text{for all}\:\:w^{\prime}\in \Gamma^*\:\:\text{iff}\:\:(\langle g,\gamma\rangle,\langle g^{\prime},w\rangle)\in \Delta\]
        \item Let $\rightarrow^*_{\Delta}$ denote the reflexive-transitive closure of $\rightarrow_{\Delta}$
        \item By definiton of $\Delta$ above, there are no transitions from a configuration whose stack is empty.
        Hence, the PDS would halt in this case.
    \end{itemize}
\end{frame}
\begin{frame}{Reachable configurations}
   \begin{itemize}
    \item A configuration $c$ is called \textit{reachable} iff $c_{in}\rightarrow_{\Delta}^* c$, where 
    $c_{in}$ is the initial configuration.
    \item A \textit{pushdown store automaton} $A=(Q,\Gamma,\delta, I,F)$ is a finite automaton with states 
    $Q$, alphabet $\Gamma$, $\delta\subseteq Q\times\Gamma\times Q$ as the transition function, initial states $I$ and
    final states $F$.
    \item The sets $Q$ and $I$ satisfy $G\subseteq Q$ and $I\subseteq G$.
    \item $A$ accepts a pushdown configuration $\langle g,w\rangle$ iff $A$ accepts $w$ when started in the state $g$.   
    \item A subset $S\subseteq G\times \Gamma^*$ is called regular if there exists a pushdown store automaton 
    $A$ such that $S=L(A)$.
    \item The reachability problem for pushdown systems is decidable because the set of reachable configurations
    from any initial configuration forms a regular set.  
   \end{itemize} 
\end{frame}
\begin{frame}{Reachable configurations (continued)}
    \begin{itemize}
        \item For a pushdown system $P= (G,\Gamma,\Delta,g_{in},w_{in})$ and a set of configurations 
        $S\subseteq G\times \Gamma^*$, $Post_{\Delta}^*(S)$ is the set of configurations reachable from $S$ i.e.
        \[Post_{\Delta}^*(S)=\{c\:|\:\exists c^{\prime}\in S,\:\:c^{\prime}\rightarrow_{\Delta}^*c\}\]
        \item The set of configurations reachable from a regular set forms a regular set. More precisely,
        \begin{theorem}\label{thm1}
            Let $P=(G,\Gamma,\Delta, g_{in},w_{in})$ be a pushdown system and let $A$ be a regular pushdown store
            automaton. There exists a regular pushdown store automaton $A^{\prime}$ such that $Post_{\Delta}^*(L(A))=L(A^{\prime})$.
            The automaton $A^{\prime}$ can be constructed from $A$ in time $O(|P|^2|A|+|P|^3)$ \cite{schwoon2002model}.
        \end{theorem}
    \end{itemize}
\end{frame}
\begin{frame}{Concurrent Pushdown Systems}
    \begin{itemize}
        \item A \textit{concurrent pushdown system} is a tuple 
        \[P=(G,\Gamma,\Delta_0,\dots,\Delta_N,g_{in},w_{in})\]
        \item $\Delta_0,\dots ,\Delta_N$ are transition relations over $G$ and $\Gamma$, $g_{in}$ is an initial 
        state and $w_{in}$ is an initial stack.
        \item A configuration of a concurrent pushdown system is a tuple $c=\langle g,w_0,\dots,w_N\rangle$ where $g\in G$ and $w_i\in\Gamma^*$.
        \item The initial configuration of $P$ is $\langle g,w_{in},\dots,w_{in}\rangle$ where all $N+1$ stacks are 
        initialized to $w_{in}$.
        \item The transition system of $P$ ($\rightarrow_P$) rewrites the global state while changing any one
        of the stacks, according to the transition relation of the PDS assciated to that stack.
        \item More formally, 
        \[\langle g,w_0,w_1\dots ,w_i,\dots w_N\rangle\rightarrow_i \langle g^{\prime},w_0,w_1\dots ,w^{\prime}_i,\dots w_N\rangle\:\:\text{iff}\] 
        \[\langle g,w_i\rangle\rightarrow_{\Delta_i}\langle g^{\prime},w^{\prime}_i\rangle\]
    \end{itemize}
\end{frame}
\begin{frame}{Concurrent Pushdown Systems (continued)}
     \begin{itemize}
        \item $\rightarrow_P$ is defined on the configurations of $P$ by the union of $\rightarrow_i$ i.e.
        \[\rightarrow_P\:=\:\bigcup_{i=0}^N \rightarrow_i\]
        \item $\rightarrow^*_P$ denotes the reflexive, transitive closure of $P$.    
        
     \end{itemize} 
\end{frame}
\begin{frame}{Reachability}
    \begin{itemize}
        \item A configuration $c$ is called reachable iff $c_{in}\rightarrow_P^* c$, where $c_{in}$ is the 
        initial configuration.
        \item In general, checking reachability for concurrent pushdown systems is an undecidable problem \cite{ramalingam2000context}.
        \item However, bounding the number of context switches leads to a decidable restriction of the problem.
        \item For $k\in \mathbb{N}$, we define $\xrightarrow{k}$ as the \textit{$k$-bounded transition relation},\\
        $c\xrightarrow{1} c^{\prime}$ iff there exists $i$ such that $c\rightarrow_i^* c^{\prime}$\\
        $c\xrightarrow{k+1} c^{\prime}$ iff there exists $c^{''}$ and $i$ such that $c\xrightarrow{k}c^{''}$ and $c^{''}\rightarrow_i^* c^{\prime}$
        \item A $k$-bounded transition contains atmost $k-1$ context switches in which a new 
        relation $\rightarrow_i$ can be chosen.
        \item A configuration $c$ is \textit{$k$-reachable} if $c_{in}\xrightarrow{k} c$. 
        \item We define the \textit{$k$-bounded reachability} problem as follows,\\
        Given two configuration $c_0$ and $c_1$, is it the case that $c_0\xrightarrow{k} c_1$?
    \end{itemize}
\end{frame}
\begin{frame}{Reachability (continued)}
    \begin{itemize}
        \item We will show that the $k$-bounded reachability problem is decidable.
        \item To do this, we define a transition relation over \textit{aggregate configurations} i.e. 
        over configurations of the form $\langle\langle g,R_0,\dots, R_N\rangle\rangle$ where $R_i$ are regular subsets 
        of $\Gamma^*$.
        \item For $g\in G$ and a regular set $R\subseteq \Gamma^*$,
        \[\langle\langle g,R\rangle\rangle\:=\:\{\langle g,w\rangle\:|\:w\in R\}\]
        \item Let $G=\{g_1,\dots,g_m\}$. Any regular set $S\subseteq G\times \Gamma^*$ can be written as 
        \begin{equation}\label{eq:1} S\:=\:\biguplus_{i=1}^{m}\langle\langle g_i,R_i\rangle\rangle\end{equation}
        where each set $R_i\subseteq \Gamma^*$ is regular.
        \item By Theorem \ref{thm1}, the set $Post^*_{\Delta}(S)$ can be written in the form (\ref{eq:1}), if $S$ is regular.  
    \end{itemize}
\end{frame}
\begin{frame}{Reachability (continued)}
    \begin{itemize}
        \item We say $\langle\langle g^{\prime},R^{\prime}\rangle\rangle\in Post^*_{\Delta}(S)$ if $Post^*_{\Delta}(S)=\biguplus_{i=1}^m \langle\langle g_i,R_i\rangle\rangle$,
        where $\langle\langle g^{\prime},R^{\prime}\rangle\rangle=\langle\langle g_j,R_j\rangle\rangle$ (for some $j$ such that $1\leq j\leq m$).
        \item We define relations $\Rightarrow_i$ on aggregate configurations,
        \[\langle\langle g,R_0,\dots ,R_i,\dots ,R_N\rangle\rangle\Rightarrow_i \langle\langle g^{\prime},R_0,\dots R^{\prime}_i,\dots,R_N\rangle\rangle\]
        if $\langle\langle g^{\prime},R^{\prime}_i\rangle\rangle\in Post_{\Delta}^*(\langle\langle g,R_i\rangle\rangle)$.
        \item We define the transition relation on aggregate configurations $\Rightarrow$ as $\bigcup_{i=0}^N\Rightarrow_i$.
        \item For aggregate configurations $a_1$ and $a_2$, $a_1\xRightarrow{k}a_2$, if $a_2$ can be reached from 
        $a_1$ using at most $k$ transitions.
        \item Using the notations described above, the following reduces
        $k$-bounded reachability problem to sequential appications of $Post_{\Delta}^*$ operator,
    \end{itemize}
\end{frame}
\begin{frame}{Reachability (continued)}
    \begin{theorem}\label{thm2}
        Let $P=(G,\Gamma,\Delta_0,\dots,\Delta_N,g_{in},w_{in})$ be a concurrent pushdown system.
        Then, for any $k$, we have $\langle g,w_0,\dots,w_N\rangle\xrightarrow{k} \langle g^{\prime},w^{\prime}_0,\dots,w^{\prime}_N\rangle$ iff
        $\langle\langle g,\{w_0\},\dots,\{w_N\}\rangle\rangle \xRightarrow{k} \langle\langle g^{\prime},R^{\prime}_0,\dots,R^{\prime}_N\rangle\rangle$ 
        for some $R^{\prime}_0,\dots, R^{\prime}_N$ such that $w^{\prime}_i\in R^{\prime}_i$ for all $i$ such that $1\leq i\leq N$.
    \end{theorem}
\end{frame}
\begin{frame}{Algorithm}
        Using Theorem \ref{thm1} and Theorem \ref{thm2}, we can generate an algorithm 
        for solving the context-bounded reachability problem for concurrent pushdown systems. 
    \vskip -1ex
    \begin{figure}[h!]
    \includegraphics[scale=0.318]{Algo_WL.jpg}
    \caption{The algorithm}
    \label{fig:1}
    \end{figure}
\end{frame}
\begin{frame}{Algorithm (continued)}
    \begin{itemize}
        \item The algorithm processes a worklist WL which contains a set of items of the form 
        $(\langle g, A_0,\dots, A_N\rangle,i)$, where $g\in G$ is a global state, $A_j$ are pushdown store automata and $i\in\{0,1,\dots,k-1\}$. 
        \item Remove(WL) removes an item from the worklist and returns it
        \item Add(WL, item) adds item to the worklist 
        \item The inital pushdown store automata $A_{in}$ has initial state $g_{in}$ and accepts only 
        $w_{in}$. 
        \item Given $A_j^{\prime}=Post^*_{\Delta}(A_j)$, we construct $A_j^{\prime}$ according to 
        Theorem \ref{thm1}, such that $L(A_j^{\prime})=Post^*_{\Delta}(L(A_j))$.
        \item $G(A_j^{\prime})=\{g^{\prime}\:|\:\exists w.\langle g^{\prime},w\rangle\in L(A_j^{\prime})\}$.
        \item All pushdown store automata involved in the algorithm have atmost one start state.
        \item Given an automata $A$ and a state $g\in G$, RENAME($A,g$) returns the result of renaming the start state (if any)
        of $A$ to $g$
        \item ANONYMIZE($A,g$) returns the result of renaming all states of $G$ (except $g$) to fresh states which are not in $G$.
    \end{itemize}
\end{frame}
\begin{frame}{Algorithm (continued)}
    %cite in first para 
    The operations RENAME and ANONYMIZE are necessary for applying Theorem \ref{thm1} repeatedly
    as the construction of the pushdown store automata uses elements of $G$ as states \cite{schwoon2002model}. 
    In order to avoid errors in the construction, renaming on states of $G$ is necessary. 
    \begin{theorem}\label{thm3}
    Let $P=(G,\Gamma, \Delta_0,\dots,\Delta_N,g_{in},w_{in})$ be a concurrent pushdown system.
    For any $k$, the algorithm terminates on input $P$ and $k$, and 
    $\langle\langle g_{in},w_{in},\dots,w_{in}\rangle\rangle \xRightarrow{k}\langle\langle g^{\prime},R^{\prime}_0,\dots,R^{\prime}_N\rangle\rangle$
    iff the algorithm outputs Reach with $\langle g^{\prime},A^{\prime}_0,\dots,A^{\prime}_N\rangle\in$ Reach such that 
    $L(A_i^{\prime})=\langle\langle g^{\prime},R^{\prime}_i \rangle\rangle$ for all $i\in\{0,1,\dots,N\}$.
    \end{theorem}
    Theorem \ref{thm2} together with Theorem \ref{thm3}, imply that the algorithm solves the $k$-bounded reachability problem. 
    The next natural step is to look at the running time of the algorithm.
\end{frame}
\begin{frame}{Algorithm (continued)}
    \begin{theorem}\label{thm4}
        Given a concurrent pushdown system $P=(G,\Gamma, \Delta_0,\dots,\Delta_N,g_{in},w_{in})$ and a bound $k$,
        the algorithm in Figure \ref{fig:1} decides the $k$-bounded reachability problem in time
        $O(k^3(N|G|)^k|P|^5)$.
    \end{theorem}
We will now look at dynamic concurrent pushdown systems with additional operations, fork and join. 
\end{frame}
\begin{frame}{Dynamic Concurrent Pushdown Systems}
    \begin{itemize}
        \item To allow for dynamic fork-join parallelism, thread identifiers are to be stored as program
        variables. They are members of the set $Tid=\{0,1,2,\dots\}$.
        \item A dynamic concurrent PDS is a tuple $(GBV,GTV,LBV,LTV,\Delta,\Delta_F,\Delta_J,g_{in},\gamma_{in})$.
        \item $GBV$ is the set of all global variables containing boolean values and $GTV$ is the 
        set of all global variables containing thread identifiers.
        \item $G$ is the (infinite) set of all valuations to the global variables.
        \item $LBV$ is the set of all local variables containing boolean values $LTV$ is the set of all
        local variables containing thread identifiers.
        \item $\Gamma$ is the (infinite) set of all valuations to the local variables.
        \item $\Delta\subseteq (G\times\Gamma)\times(G\times\Gamma^*)$ is the transition relation describing the single step of any thread.
        \item $\Delta_F\subseteq Tid\times (G\times\Gamma)\times(G\times\Gamma^*)$ is the fork transition 
        relation.
    \end{itemize}
\end{frame}
\begin{frame}{Dynamic Concurrent Pushdown Systems (continued)}
    \begin{itemize}
        \item If $(t,\langle g,\gamma\rangle,\langle g^{\prime},w\rangle)\in \Delta_F$, then with the global store 
        being $g$, a thread with $\gamma$ at the top of its stack may fork a thread with identifier $t$, changing the global 
        store to $g^{\prime}$ and rewriting the top of the stack to $w$.
        \item $\Delta_J\subseteq LTV\times (G\times\Gamma)\times(G\times\Gamma^*)$ is the join transition relation.
        \item If $(x,\langle g,\gamma\rangle,\langle g^{\prime},w\rangle)\in \Delta_J$, then with the global store being $g$,
        a thread with $\gamma$ at the top of its stack blocks until the thread with identifier $\gamma(x)$ 
        finishes execution. On getting unblocked, the global store is modified to $g^{\prime}$ and the top
        of the stack is rewritten with $w$.
        \item $g_{in}$ is the initial valuation to the set of all global variables such that 
        $g_{in}(x)=0$ for all $x\in GTV$. 
        \item $\gamma_{in}$ is the initial valuation to the set of all local variables such that 
        $\gamma_{in}(x)=0$ for all $x\in LTV$. 
    \end{itemize}
\end{frame}
\begin{frame}{Dynamic Concurrent Pushdown Systems (continued)}
    \includegraphics[scale=0.35]{dcpds.jpg}
    \begin{itemize}
        \item Every dynamic concurrent PDS is equipped with a special symbol $\$\not\in\Gamma$ to mark the bottom of each stack.
        \item A configuartion of the system is $\langle g,n,ss\rangle$, where $g$ is the global state, 
        $n$ is the identifier of the last thread to be forked and $ss(t)$ is the stack for the thread 
        with identifier $t\in Tid$. 
        \item The execution of the dynamic concurrent PDS starts in the configuration $\langle g_{in},0,ss_0\rangle$,
        where $ss_0(t)=\gamma_{in}\$$ for all $t\in Tid$.    
        \item The execution follows a certain set of rules which define the transitions that may be performed
        starting from any thread $t$ in the configuration $\langle g,n,ss\rangle$. 
    \end{itemize}
\end{frame}
\begin{frame}{Operational Semantics}
   \includegraphics[scale=0.255]{op_semantics.jpg} 
   \centering
   \begin{itemize}
    \item All rules have the condition $t\leq n$, which indicates that thread $t$ must have already been forked.
    \item The rule SEQ allows thread $t$ to perform a move according to the transition relation $\Delta$.
    \item The rule SEQEND is enabled if the only symbol of the stack of thread $t$ is $\$$ which is popped from the stack 
    without changing the global state.
\end{itemize}
\end{frame}
\begin{frame}{Operational Semantics (continued)}
    \begin{itemize}
        \item The rule FORK creates a new thread with identifier $n+1$. 
        \item The rule JOIN is enabled if thread $\gamma(x)$ has terminated, where $\gamma$ is the top of the
        stack of thread $t$. Termination of a thread is indicated by an empty stack.
    \end{itemize}
\end{frame}
\begin{frame}{Assumptions}
    \begin{itemize}
        \item In realistic concurrent programs, the usage of thread identifiers is restricted.
        \item In view of this, we introduce some assumptions. 
        \item A \textit{renaming function} is a partial function from $Tid$ to $Tid$. 
        \item When a renaming function is applied to a global store $g$, it returns another store in which 
        every variable of type $Tid$ is transformed by an application of $f$. 
        \item If $f$ is undefined on some global variable in $g$, it is undefined on $g$.
        \item A renaming function can also be applied to a local store or a sequence of local stores by a pointwise 
        application to each element of the sequence.
    \end{itemize}
\end{frame}
\begin{frame}{Assumptions (continued)}
    \textbf{A1.} For all $g\in G, \gamma\in \Gamma$ and renaming functions $f$ such that $f(g)$ and $f(\gamma)$ are defined,
    \begin{enumerate}
        \item If $(\langle g,\gamma\rangle,\langle g^{\prime},w\rangle)\in \Delta$, then there exists 
        $fg^{\prime}\in G$ and $fw\in\Gamma^*$ such that $fg^{\prime}=f(g^{\prime})$, $fw=f(w)$, and
        $(\langle f(g),f(\gamma)\rangle,\langle fg^{\prime},fw\rangle)\in \Delta$.
        \item If $(\langle f(g),f(\gamma)\rangle,\langle fg^{\prime},fw\rangle)\in \Delta$, then there 
        exists $g^{\prime}\in G$ and $w\in \Gamma^*$ such that $fg^{\prime}=f(g^{\prime})$, $fw=f(w)$ and 
        $(\langle g,\gamma\rangle,\langle g^{\prime},w\rangle)\in \Delta$.
    \end{enumerate}
    \textbf{A2.} For all $t\in Tid$, $g\in G$, $\gamma\in \Gamma$ and renaming functions $f$ such that 
    $f(t)$, $f(g)$ and $f(\gamma)$ are all defined,
    \begin{enumerate}
        \item If $(t,\langle g,\gamma\rangle,\langle g^{\prime},w\rangle)\in \Delta_F$, then there exists 
        $fg^{\prime}\in G$ and $fw\in \Gamma^*$ such that $fg^{\prime}=f(g^{\prime})$, $fw=f(w)$ and 
        $(f(t),\langle f(g),f(\gamma)\rangle,\langle fg^{\prime},fw\rangle)\in\Delta_F$.
        \item If $(f(t),\langle f(g),f(\gamma)\rangle,\langle fg^{\prime},fw\rangle)\in \Delta_F$, then there 
        exists $g^{\prime}\in G$ and $w\in \Gamma^*$ such that $fg^{\prime}=f(g^{\prime})$, $fw=f(w)$ and 
        $(t,\langle g,\gamma\rangle,\langle g^{\prime},w\rangle)\in \Delta_F$.
    \end{enumerate} 
\end{frame}
\begin{frame}{Assumptions (continued)}
    \textbf{A3.} For all $x\in LTV$, $g\in G$, $\gamma\in \Gamma$ and renaming functions $f$ such that 
    $f(g)$ and $f(\gamma)$ are defined,
    \begin{enumerate}
        \item If $(x,\langle g,\gamma\rangle,\langle g^{\prime},w\rangle)\in \Delta_J$, then there exists 
        $fg^{\prime}\in G$ and $fw\in \Gamma^*$ such that $fg^{\prime}=f(g^{\prime})$, $fw=f(w)$ and 
        $(x,\langle f(g),f(\gamma)\rangle,\langle fg^{\prime},fw\rangle)\in\Delta_J$.
        \item If $(x,\langle f(g),f(\gamma)\rangle,\langle fg^{\prime},fw\rangle)\in \Delta_J$, then there 
        exists $g^{\prime}\in G$ and $w\in \Gamma^*$ such that $fg^{\prime}=f(g^{\prime})$, $fw=f(w)$ and 
        $(x,\langle g,\gamma\rangle,\langle g^{\prime},w\rangle)\in \Delta_J$.
    \end{enumerate}
    Henceforth, these are the assumptions we will be making on $\Delta$, $\Delta_F$ and $\Delta_J$.\\
    By making these asssumptions, we are exploiting the restricted usage of thread identifiers in concurrent systems.
\end{frame}
\begin{frame}{Reduction to Concurrent PDS}
    \begin{itemize}
        \item We will reduce the \textit{$k$-bounded reachability} problem on a dynamic concurrent
        PDS to a concurrent PDS with $k+1$ threads. 
        \item Given a dynamic concurrent PDS $P$ and a positive integer $k$, we will construct a 
        concurrent PDS $P_k$ containing $k+1$ threads with identifiers in $\{0,1,\dots,k\}$ such that 
        it suffices to verify the $k$-bounded executions of $P_k$.
        \item In a $k$-bounded execution, at most $k$ different threads may perform a transition. 
        \item The last thread in $P_k$ doesn't perform a transition, it only exists to simulate
        the remaining threads in $P$. 
        \item Let $Tid_k=\{0,1,\dots, k\}$ be the set of thread identifiers bounded by $k$. 
        \item $AbsG_k$ and $Abs\Gamma_k$ are finite sets which consist of all valuations to global and local variables, 
        where the variables of thread identifier type are assigned values from $Tid_k$.  
    \end{itemize}
\end{frame}
\begin{frame}{Reduction to Concurrent PDS (continued)}
    \begin{itemize}
        \item Given a dynamic concurrent PDS $P=(GBV,GTV,LBV,LTV,\Delta,\Delta_F,\Delta_J,g_{in},\gamma_{in})$, we define,
        \[ P_k=(AbsG_k\times Tid_k\times \mathscr{P}(Tid_k),Abs\Gamma_k\cup \{\$\}, \Delta_0,\dots,\Delta_k,(g_{in},0,\emptyset),\gamma_{in}\$)\]
        \item $P_k$ has $k+1$ threads.
        \item A global state of $P_k$ is a three-tuple $(g,n,\alpha)$, where $g$ is a valuation to the global variables,
        $n$ is the largest thread identifier whose corresponding thread is allowed to make a transition and 
        $\alpha$ is the set of thread identifiers whose stacks are empty.
        \item The initial state is $(g_{in},0,\emptyset)$ which means that only thread $0$ is allowed to make 
        a transition and no thread has finished execution.
        \item There are rules for every transition relation $\Delta_t$, for each thread $t\in Tid_k$.
    \end{itemize}
\end{frame}
\begin{frame}{Reduction to Concurrent PDS (continued)}
    \includegraphics[scale=0.3]{Delta_t_defn.jpg}
    \begin{itemize}
        \item All transitions are guarded by the condition $t\leq n$ to ensure that a thread $t$ cannot
        make a transition if $t>n$.
        \item The rule ABSSEQ adds transitions from $\Delta$ to $\Delta_t$. 
        \item The rule ABSSEQEND adds thread $t$ to the set of terminated threads. 
        \item The rules ABSFORK and ABSFORKNONDET handle thread creation in $P$.  
    \end{itemize}
\end{frame}
\begin{frame}{Reduction to Concurrent PDS (continued)}
    \begin{itemize}
        \item The rule ABSFORK increments the counter $n$ allowing thread $n+1$ to simulate the newly
        forked thread which participates in the $k$-bounded execution.
        \item The rule ABSFORKNONDET leaves the counter unchanged and the newly forked thread doesn't participate 
        in the $k$-bounded execution.
        \item The rule ABSJOIN adds rules from $\Delta_J$ to $\Delta$ by using the fact that the identifiers
        of all previously terminated threads are present in $\alpha$.
        \item We will now state the correctness theorems for the transformation.\\ A configuration of 
        $P_k$, $\langle(g^{\prime},n^{\prime},\alpha),w_0,w_1,\dots,w_k\rangle$, can be written as 
        $\langle(g^{\prime},n^{\prime},\alpha),ss^{\prime}\rangle$, where $ss^{\prime}$ is a map from $Tid_k$ to $(Abs\Gamma_k\cup \$)^*$. 
    \end{itemize}
\end{frame}
\begin{frame}{Correctness theorems}
    \begin{theorem}[Soundness]\label{thm5}
       Let $P$ be a dynamic concurrent pushdown system and $k$ be a positive integer. 
       Let $\langle g,n,ss\rangle$ be a $k$-reachable configuration of $P$. Then there is a total 
       renaming function $f: Tid\rightarrow Tid_k$ and a $k$-reachable configuration $\langle(g^{\prime},n^{\prime},\alpha),ss^{\prime}\rangle$
       of the concurrent pushdown system $P_k$ such that $g^{\prime}=f(g)$ and $ss^{\prime}(f(j))=f(ss(j))$ for all $j\in Tid$.
    \end{theorem}
    \begin{theorem}[Completeness]\label{thm6}
        Let $P$ be a dynamic concurrent pushdown system and $k$ be a positive integer. 
        Let $\langle(g^{\prime},n^{\prime},\alpha),ss^{\prime}\rangle$ be a $k$-reachable configuration of the concurrent 
        pushdown system $P_k$. Then there is a total renaming function $f:Tid\rightarrow Tid_k$ and a 
        $k$-reachable configuration $\langle g,n,ss\rangle$ of $P$ such that $g^{\prime}=f(g)$ and $ss^{\prime}(f(j))=f(ss(j))$ for all $j\in Tid$.
    \end{theorem}
 With Theorems \ref{thm5} and \ref{thm6}, we can conclude that our reduction from a dynamic 
 concurrent pushdown system to a concurrent pushdown system with $k+1$ threads is correct. 
\end{frame}
\begin{frame}{Related Work}
    %cite here
    %paper ke introduction se hi kar rahe hain, hopefully it's fine
    \begin{itemize}
        \item The classical notion of context bounding is too restrictive.
        \item For dynamic systems, bounding the number of context switches bounds the number of 
        threads involved.
        \item $K$-bounded computation: Each thread can be interrupted and resumed at most $K$ times. 
        \item Decidability also holds under this new restriction.
        \item Given a program $P$ and an integer $K$, the problem of checking whether a program fails
        some assertion in a $K$-bounded computation is EXPSPACE-complete \cite{atig2011context}.
        %\item Under the classical notion of context-boundedness, \textit{$k$-bounded reachability} 
        %is an NP-complete. 
    \end{itemize}
\end{frame}
\begin{frame}[allowframebreaks]{References}
    \bibliography{check}{}
    \bibliographystyle{plain}
\end{frame}
\begin{frame}{Conclusion}
    \begin{center}
        THANK YOU!
    \end{center}
\end{frame}
\end{document}