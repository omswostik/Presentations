\documentclass[10pt]{beamer}
\usetheme[progressbar=frametitle]{metropolis}
\usepackage{appendixnumberbeamer}

\usepackage{booktabs}
\usepackage[scale=2]{ccicons}
\usepackage{amsmath}
\usepackage{amsthm}
\usepackage{amssymb}
\usepackage{tikz}
\usepackage{graphicx}
\graphicspath{{C:\Users\swost\Gdb\PA_try-2.png}}
\usepackage{pgfplots}
\usepgfplotslibrary{dateplot}
\usepackage{cite}
\usepackage{todonotes}
\usepackage{upgreek}
\usepackage{bm}
\usepackage{xspace}
\newcommand{\themename}{\textbf{\textsc{metropolis}}\xspace}
\usepackage{subcaption}

\definecolor{CanvasBG}{HTML}{FAFAFA}

% From the official style guide
\definecolor{UnccGreen}{HTML}{00703C}
\definecolor{UnccGold}{HTML}{B3A369}
\definecolor{UnccLightGreen}{HTML}{C3D7A4}
\definecolor{UnccYellow}{HTML}{F0CB00}
\definecolor{UnccOrange}{HTML}{F3901D}
\definecolor{UnccLightYellow}{HTML}{FFF6DC}
\definecolor{UnccBlue}{HTML}{00728F}
\definecolor{UnccPink}{HTML}{DE3A6E}
\definecolor{White}{HTML}{FFFFFF}
\definecolor{LightGray}{HTML}{DDDDDD}

\definecolor{WarmGray}{HTML}{696158}
\definecolor{StoneGray}{HTML}{717C7D}
\definecolor{DarkGreen}{HTML}{2C5234}
\definecolor{LightGreen}{HTML}{509E2F}
\definecolor{BrightGold}{HTML}{F0CB00}

\definecolor{Royal}{HTML}{72246C}
\definecolor{Ocean}{HTML}{006BA6}
\definecolor{Flash}{HTML}{B52555}
\definecolor{Citrus}{HTML}{FFB81C}
\definecolor{Spring}{HTML}{CEDC00}

\definecolor{Garden}{HTML}{B7CE95}
\definecolor{Sand}{HTML}{F0E991}
\definecolor{Bloom}{HTML}{F1E6B2}
\definecolor{Clay}{HTML}{B7B09C}
\definecolor{Cloud}{HTML}{BAC5B9}

\setbeamercolor{frametitle}{bg=UnccGreen}
\setbeamercolor{progress bar}{bg=BrightGold, fg=UnccGreen}
\setbeamercolor{alerted text}{fg=Flash}

\setbeamercolor{block title}{bg=LightGreen, fg=White}
\setbeamercolor{block title example}{bg=Ocean, fg=White}
\setbeamercolor{block title alerted}{bg=Citrus, fg=White}
\setbeamercolor{block body}{bg=CanvasBG}

\metroset{titleformat=smallcaps, progressbar=foot}

\makeatletter
\setlength{\metropolis@progressinheadfoot@linewidth}{2pt}
\setlength{\metropolis@titleseparator@linewidth}{2pt}
\setlength{\metropolis@progressonsectionpage@linewidth}{2pt}

\usepackage[T1]{fontenc}
\usefonttheme{serif}
\usefonttheme{professionalfonts}
\usepackage{amsmath, tgpagella, eucal, eufrak}
\usepackage{amssymb}
\usepackage{graphicx}
\usepackage{varioref}
\usepackage{hyperref}
\usepackage[noabbrev]{cleveref}
\DeclareMathOperator*{\argmin}{arg\,min}
\DeclareMathOperator*{\argmax}{arg\,max}
\title{Presburger Arithmetic}
\subtitle{A Short Survey}
% \date{\today}
\date{}
\author{Om Swostik Mishra}
\institute{IIT Bombay}
% \titlegraphic{\hfill\includegraphics[height=1.5cm]{logo.pdf}}

\begin{document}

\begin{frame}
    \maketitle
\end{frame}

\begin{frame}{History and Introduction}
    \begin{itemize}
        \item The mathematician Mojżesz Presburger formulated the basic principles of Presburger arithmetic in his 1929 paper 
        "On the completeness of a certain system of arithmetic of whole numbers, in which addition occurs as the only operation".
        \item In this paper, it was established that Presburger arithmetic is complete and decidable.
        \item This result was achieved by using the 'quantifier elimination' method.
        \item Presburger showed the completeness of $Th(\mathbb{Z},+,0,1)$, the first order theory of integers with addition, equality and standard axioms of arithmetic.
    \end{itemize}
\end{frame}
\begin{frame}{An Example}
    \begin{itemize}
    \item Even though Presburger arithmetic captures only a fragment of number theory, lots of interesting problems can be expressed
    using it. 
    \item Given positive integers $m_1,m_2\dots m_n$, what is the largest number that cannot be obtained
    as a non-negative linear combination of those numbers?
    The answer, if it exists, is the smallest satisfying assignment of the formula:
    \end{itemize}
    \[\upphi(x) =\forall y\: (x < y \rightarrow (\exists z_1,z_2\dots z_n(y=z_1m_1+\dots z_nm_n\wedge z_1\geq 0 \wedge \dots \wedge z_n\geq 0)))\]
    \begin{itemize}
    \item Any system of linear inequalities/equations can also be expressed using Presburger arithmetic.
    \end{itemize}
\end{frame}
\begin{frame}{Quantifier elimination}
    The first approach to deciding PA is the quantifier elimination method \cite{cooper1972theorem}. 
    Here's an example to illustrate this idea:\\
    Given $\exists x,y,z \:\: (2x+4y-3z<7)\wedge (3x-y+2z<-4)$, we will eliminate the quantifier $\exists z$.
    Notice,\\ 
    $\exists x,y,z \:\: (2x+4y-3z<7)\wedge (3x-y+2z<-4) \Leftrightarrow \exists x,y,z\:\: (2x+4y-7<3z)\wedge (2z<-4+y-3x)\Leftrightarrow$\\
    $\exists x,y,z \:\: (4x+8y-14<6z)\wedge (6z<-12+3y-9x)\Leftrightarrow$ \\
    $\exists x,y \:\: (13x+5y-2<0)$\\
    For the above example, we are looking for solutions over the reals. 
    If we want to restrict our solution space to $\mathbb{Z}$, we need to introduce additional constraints:
    \[\bigvee_{1\leq m\leq 6} (6\:|\:4x+8y-14+m) \wedge (13x+5y-2+m<0)\]
    \[\vee\bigvee_{1\leq m\leq 6} (6\:|\:-9x+3y-12-m) \wedge (13x+5y-2+m<0)\]
\end{frame}
\begin{frame}{Quantifier elimination (continued)}
    We will now describe the general process for eliminating quantifiers from any given formula.
    Note that $\forall x \:F \equiv \neg \exists x \: \neg F$, hence it suffices to restrict our attention to $\exists$ quantifiers.
    Given any any formula of the form $\exists \:F$ where $F$ is quantifier-free, we proceed as follows:
    \begin{itemize}
        \item Transform $F$ to disjunctive normal form and distribute $\exists \: x$ over the disjuncts.
        We will perform elimination separately for each conjunct of relations or negations of relations.
        \item Eliminate negation by using $\neg(\alpha < \beta) \rightarrow \beta <\alpha +1$ and 
        $\neg (\delta\:|\:\alpha) \rightarrow \vee_{i=1}^{\delta-1} \delta \:|\: \alpha+i$.
        \item Simplify each relation by collecting the $x$ terms on one side. 
        If a term doesn't invlove $x$, take it outside the quantifier.
        We are left with terms of the form: $\lambda x<\alpha$, $\beta <\mu x$ and $\delta\:|\: \nu x+\gamma$.
    \end{itemize}
\end{frame}
\begin{frame}{Quantifier elimination (continued)}
    \begin{itemize}
        \item Let $\delta$ be the LCM of $\alpha$, $\nu$ and $\beta$ (i.e coefficients of $x$) over all relations in the conjunct. 
        Multiply both sides of all relations with appropriate constants such that the coefficients of all $x$'s are made $\delta$.
        Replace $\exists x\: F(\delta x)$ with $\exists x\: (F(x)\wedge \delta\:|\: x)$. The result will again be a conjunct of 
        relations but the coefficient of $x$ in every term is $1$.
        \item The elimination is performed using the equivalence:
        \[ \exists x \:(\alpha<x\wedge  x<\beta \wedge  \delta\: |\: x) \equiv \bigvee_{j=1}^{\delta} (\alpha+j<\beta \wedge \delta\:|\:\alpha+j) \] 
    \end{itemize}
    It has been shown that this algorithm runs in deterministic triply exponential time \cite{oppen1978222pn}.
\end{frame}
\begin{frame}{Automata-conversion}
    \begin{itemize}
        \item The aim is to construct an automaton whose language encodes all satisfying assignments of the given PA formula.
        \item Every formula in Presburger arithmetic can be interpreted as a Monadic Second Order formula on the integers with the $<$ relation
        and a $+$ function.
        \item Given any MSO formula, there exists a translation to a finite state automaton 
        such that the language accepted by this automaton is exactly the set of satisfying assignments of the formula.
        \item This approach can lead to high complexity due to the possibility of repeated complementation. 
        However, it has been shown that PA formulas have a special structure that prevents this non-elementary blow-up from happening \cite{durand2010use}.
        \item The run-time of this automata-based construction is triply-exponential in the size of the input formula. 
        This is also a optimal bound. 
    \end{itemize}
\end{frame}
\begin{frame}{Automata-conversion (Example)}
   \begin{figure}[h!]
   \includegraphics[scale=0.55]{PA_try-2.png}
   \caption{Finite automaton encoding the satisfying assignments of $\Phi$}
   \end{figure}
   Let $\Phi(x,y,z)= (+(x,y)=z)$. The above figure represents an automata which accepts a tuple $(i,j,k)$ iff $i+j=k$. 
   \begin{itemize}
   \item The automata reads the binary representation of the numbers, adds digit by digit 
   and accepts at $0$ as long as a carry doesn't occur. 
   \item When a carry does occur, the automata switches to state $1$ and stays there until the carry has been resolved.
   \item Once the carry is resolved, the automata switches back to state $0$ and accepts.
\end{itemize}
\end{frame}
\begin{frame}{Semi-Linear sets}
    %\vskip-1ex
    \begin{itemize}
        %\addtolength{\itemsep}{-0.5ex}
        \item Given a base vector $b\in \mathbb{Z}^d$ and a finite set of period vectors $P=\{p_1,\dots ,p_n\}\subseteq \mathbb{Z}^d$,
        the linear set $L(b,P)$ is defined as 
        \[ L(b,P)=b+\{\lambda_1p_1+\dots +\lambda_np_n\:: \:\lambda_1\geq0\wedge \dots \wedge \lambda_n\geq 0\}\]
        \item A semi-linear set is a finite union of linear sets.
        \item Semi-linear sets are trivially closed under projection.
    \end{itemize}
\end{frame}
\begin{frame}{Semi-Linear sets (continued)}
    \begin{itemize}
        \item Every linear set is definable in Presburger arithmetic. 
        Let $v(i)$ denote the $i$-th component of vector $v$. $x\in L(b,P)$ iff $x$ is a solution of
        \[\Phi(x)=\exists \lambda_1\dots \lambda_n\:\bigwedge_{1\leq x\leq d} x(i)=b(i)+\lambda_1p_1(i)+\dots+\lambda_np_n(i)\]
        \item Since a semi-linear set is a finite union of linear sets, it is also definable in PA.
        \item The reverse also holds i.e. every set of integer tuples definable in PA forms a semi-linear set.
        \item Since PA admits quantifier elimination, it suffices to show that the set of solutions to a system of linear
        inequalities and a system of linear congruences is semi-linear and that semi-linear sets are closed under intersection.
    \end{itemize}
\end{frame}
\begin{frame}{Systems of linear equations}
    \begin{itemize}
        \item Given a $d\times n$ matrix $A$, we will show that the set $S\subseteq \mathbb{N}_0^n$ consisting of non-negative 
        integer solutions to $Ax=0$ is semi-linear.
        \item $S$ is a commutative monoid with respect to addition.   
        \item Given vectors $v,w\in S$ with non-negative entries, we define a partial ordering $<$ such that $v\leq w$ if $v(i)\leq w(i)$, 
         $\forall i$. 
        \item With respect to the ordering $<$, the set $S$ has finitely many minimal elements (follows from Dickson's lemma). 
        \item Let $P\subset S$ be the set of all minimal elements. It can be shown (by induction) that $P$ generates every element of
        $S$.  
        \item We can conclude that $S=L(0,P)$, which is a linear set.
    \end{itemize}
\end{frame}
\begin{frame}{Linear equations (continued)}
    \begin{itemize}
        \item Consider the non-homogenous case i.e. let $S\subseteq \mathbb{N}_0^n$ be the set of all solutions of the 
        equation $Ax=b$. 
        \item If $S$ is empty i.e. the equation has no solution, then semi-linearity of $S$ trivially follows.
        \item If $S$ is non-empty, consider the set $B$ consisting of all minimal elements of $S$.
        Finiteness of $B$ follows from Dickson's lemma.
        \item Let $P$ be the finite set of minimal vectors which generates all solutions of the homogenous equation $Ax=0$.
        \item Using the same argument as for the homogenous case, it follows,
        \[ S=L(B,P)=\bigcup_{b\in B}L(b,P) \]  
    \end{itemize}
\end{frame}
\begin{frame}{Linear inequalities}
    \begin{itemize}
    \item Given a system of linear inequalities $Ax\geq b$, let $S\subseteq \mathbb{N}_0^n$ denote the set of solutions,
    which we will show to be semi-linear.
    \item Let $B\subset S$ denote the set consisting of minimal elements of $S$. Finiteness of $B$ follows from 
    Dickson's lemma. 
    \item Let $P$ denote the finite set of minimal vectors satisfying the inequality $Ax\geq 0$. 
    As before, we can show that, the set of vectors in $P$ generate all solutions of the system $Ax\geq 0$.
    \item Using the same argument as was done for the case of linear equations, we have,
    \[ S=L(B,P)=\bigcup_{b\in B}L(b,P) \]
\end{itemize}
\end{frame}
\begin{frame}{Linear congruences}
    \begin{itemize}
    \item Given a system of divisibility constraints, the set of solutions $S\subseteq \mathbb{N}_0^n$ forms a semi-linear set.
    \item Consider the system 
     \[\Phi(x) = \bigwedge_{1\leq i\leq d} c_i\: |\: p_i(x)\]
     where $x=(x_1,\dots,x_n)$ and each 
     $p_i(x)$ is a linear expression in $x_1,\dots,x_n$.
    \item Let $c=lcm(c_1,\dots,c_n)$ 
    and $B=\{v\in \{0,1\dots,c-1\}^n\:|\:\Phi(v/x)\: \text{is true}\}$.
    \item Now, let $P=\{c.e_i\:|\:1\leq i\leq n\}$, where $e_i$ is the $i$-th unit vector.
    \item $S=L(B,P)$ is the set of non-negative integer solutions of $\Phi(x)$.
    \end{itemize}
\end{frame}
\begin{frame}{Closure under intersection}
    \begin{itemize}
    \item Semi-linear sets are closed under intersection.
    \item Due to distributivity of union and intersection, it suffices to show intersection
    of two linear sets is semi-linear.
    \item Let $L(c,Q)$ and $L(d,R)$ be linear sets. \\
    $v\in L(c,Q)\cap L(d,R)$ iff there exists $\lambda,\gamma\geq 0$ such that, 
    \begin{align*}
        &v=c+Q\lambda\: \:\text{and}\:\:v=d+R\gamma\\
      \Leftrightarrow\:\:&c+Q\lambda=d+R\gamma\\
      \Leftrightarrow\:\:&(Q\:|-R)\begin{bmatrix}\lambda \\ \gamma \end{bmatrix}= d-c
    \end{align*}
    \item The last condition reduces to a system of linear equations which we have shown to be semi-linear.
    Let $L(E,S)$ be the semi-linear set obtained after projecting the solution set to the $\lambda$-coordinates.
\end{itemize}
\end{frame}
\begin{frame}{Closure under intersection (continued)}
    \begin{itemize}
        \item Let $B= c+ QE$ and $P=QS$. Recall that $v= c+Q\lambda$, where $\lambda\geq 0$.
        \begin{align*}
            L(c,Q)\cap L(d,R) &= c+\{Qw\:|\:w\in L(E,S)\}\\
            &=c+Q.L(E,S)\\
            &=c+Q.\{E+S\zeta\:|\:\zeta\geq 0\}\\
            &=L(c+ Q.E,P=Q.S)\\
            &=L(B,P)
        \end{align*}
        \item It follows that semi-linear sets are closed under intersection.
        \item We can conclude, given any PA formula, its set of solutions forms a semi-linear set.
    \end{itemize}
\end{frame}
\begin{frame}{Decomposition of semi-linear sets}
    \begin{itemize}
        \item A linear set $L(b,P)$ decomposes as 
        \[ L(b,P)=\bigcup_{i\in I}L(b_i,P_i)\] 
        where $P_i\subseteq P$ are linearly independent \cite{ginsburg1963bounded}.
        \item We have an even stronger property \cite{ito1969every}. Every semi-linear set $M$ is equivalent
        to a semi-linear set 
        \[M= \bigcup _{i\in I} L(b_i,P_i)\]
        such that all $P_i$ are linearly independent and $L(b_i,P_i)\cap L(b_j,P_j)=\emptyset$, $\forall\:i\neq j$.
    \end{itemize}
\end{frame}
\begin{frame}{Complementation}
    \begin{itemize}
        \item Complement of a semi-linear set also forms a semi-linear set.
        \item This follows from the equivalence between semi-linear and Presburger definable sets.
        \item Alternatively, we can give a direct proof of this result. 
        \item Due to closure under union and intersection, it suffices to show complementation
        of a linear set is semi-linear.
        \item Let $M=L(b,P)$ be a linear set and WLOG assume $P$ is linearly independent.
        \item Let $\widetilde{M}$ denote the convex hull of $M$. 
        \item By Minkowski-Weyl's theorem, there is a system of linear inequalities $Ax\leq c$ defining $\widetilde{M}$.
        \item Then the set $\mathbb{R}^d\setminus \widetilde{M}$ can be obtained as the set of all solutions of the system 
        $Ax>c$.
    \end{itemize}
\end{frame}
\begin{frame}{Complementation (continued)}
    \begin{itemize}
        \item For every $v\in M$, there exists unique $\lambda\in \mathbb{N}^n$ such that
        $v=b+P\lambda$. 
        \item Any $w\in \widetilde{M}\setminus M$ can be obtained as $w=b+P\gamma$
        with the exception that some component of $\gamma$ is not integral.
        \item Define $C=b+(\{v\in P\lambda\cap \mathbb{Z}^d\::\:\lambda \in [0,1)^n\}\setminus \{0\})$
        which gives, $\widetilde{M}\setminus M= L(C,P)$.
        \item We can now realize, 
        \[\mathbb{Z}^d\setminus M=((\mathbb{R}^d\setminus \widetilde{M})\cap \mathbb{Z}^d)\cup L(C,P)\] 
        as a semi-linear set.
    \end{itemize}
\end{frame}
\begin{frame}{Descriptional Complexity}
    \begin{itemize}
        \item The aim is to keep track of the constants and the size of the generator set
        while describing a semi-linear set.
        \item Let $Ax=0$ be a homogenous system and consider the set of its non-negative integer solutions,
        with the generator set $P$. 
        \item  If $\|P\|$ denotes the largest absolute value in $P$, 
        \[\|P\|\leq (1+\|A\|_{1,\infty})^r\]
        \item This bound was obtained by analyzing minimal solutions of linear diophantine systems \cite{pottier1991minimal}.
    \end{itemize}
\end{frame}
\begin{frame}{Computational Complexity}
    \begin{itemize}
        \item Cooper's quantifier elimination algorithm runs in deterministic triply exponential time \cite{oppen1978222pn}.
        \item In 1974, a non-deterministic doubly exponential time lower bound was shown for full
        Presburger arithmetic \cite{fischer1998super}. This was the first hardness result for PA.
        \item In 1980, Berman \cite{berman1980complexity} showed that Presburger arithmetic is complete for $STA(*,2^{2^p(n)},O(n))$, where $p$ 
        is a fixed polynomial.
        \item The above result yields a doubly exponential space upper bound.
        \item The high lower bounds for Presburger arithmetic require formulas
        with an unbounded number of alternations. Fixing the number of quantifiers/alternations 
        lowers the complexity. \nocite{haase2018survival}
    \end{itemize}
\end{frame}
%\begin{frame}{Computational Complexity (continued)}
%    \begin{itemize}
%        \item 
%    \end{itemize}
%\end{frame}
\begin{frame}[allowframebreaks]{References}
\bibliography{check}{}
\bibliographystyle{plain}
\end{frame}
\begin{frame}{Conclusion}
    \begin{center}
        THANK YOU!
    \end{center}
\end{frame}
\end{document}